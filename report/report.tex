% Created 2014-06-02 mån 18:24
\documentclass[11pt]{article}
\usepackage[utf8]{inputenc}
\usepackage[T1]{fontenc}
\usepackage{fixltx2e}
\usepackage{graphicx}
\usepackage{longtable}
\usepackage{float}
\usepackage{wrapfig}
\usepackage{rotating}
\usepackage[normalem]{ulem}
\usepackage{amsmath}
\usepackage{textcomp}
\usepackage{marvosym}
\usepackage{wasysym}
\usepackage{amssymb}
\usepackage{hyperref}
\tolerance=1000
\usepackage[parfill]{parskip}
\usepackage{mathtools}
\usepackage[utf8]{inputenc}
\usepackage[swedish, english]{babel}
\usepackage[T1]{fontenc}
\usepackage{moreverb,fancyheadings,graphicx, amssymb}
\usepackage{fixltx2e}
\usepackage{longtable}
\usepackage{float}
\usepackage{wrapfig}
\usepackage{soul}
\usepackage{textcomp}
\usepackage{marvosym}
\usepackage{wasysym}
\usepackage{latexsym}
\usepackage{hyperref}
\author{Anton Erholt \& Christopher Mårtensson \\ <aerholt@kth>}
\date{\today}
\title{Parallelize particle simulation}
\hypersetup{
  pdfkeywords={},
  pdfsubject={A project in the course ID1217 at KTH},
  pdfcreator={Emacs 24.3.1 (Org mode 8.2.6)}}
\begin{document}

\maketitle
\newpage
\begin{abstract}

This report serves to describe a programming project in the course ID1217,
Concurrent programming. The project was to implement an algorithm for particle
simulation which ran in time close to $O(n)$.

\end{abstract}
\newpage


\section*{Introduction}
\label{sec-1}

\section*{The algorithm}
\label{sec-2}

\section*{Implementation details}
\label{sec-3}

\section*{Calculations and results}
\label{sec-4}

\section*{Discussion and thoughts}
\label{sec-5}

\section*{Plots and figures}
\label{sec-6}
% Emacs 24.3.1 (Org mode 8.2.6)
\end{document}